%---------- Inleiding ---------------------------------------------------------
\section{Inleiding}
\label{sec:inleiding}
% TODO: concrete research question
Bij het opstarten van een IMS-subsysteem zijn er drie soorten die door IBM worden aangeboden: \emph{Warm start}, \emph{Cold start} en \emph{Emergency restart}. Binnen KBC heeft het Automation-team 
(subteam binnen Mainframe infrastructuur) uniforme start- en stopprocedures voorzien voor een groot aantal situaties. Er zijn ook een aantal gespecialiseerde procedures ontwikkeld die niet in de IBM-handleidingen te vinden zijn. 
Er moet dus een onderscheid worden gemaakt tussen wat IBM heeft voorzien – de drie hierboven beschreven soorten opstartprocedures – en wat via Automation in KBC is ingebouwd. 
Elke procedure is gebaseerd op het opstarten van het IMS-(sub)systeem, maar er worden ook aanvullende controles en commando's uitgevoerd, zoals het opstarten van regio's, het opstarten van IMS-verbindingen, enz.\\

Het IMS Infra-team is verantwoordelijk voor het beheer van het opstarten (en afsluiten) en dit is gebaseerd op een niveaustructuur. Elk niveau heeft een uniek nummer en voert een specifieke taak uit bij het opstarten. 
Elk niveau is functioneel hetzelfde in de verschillende IMS-systemen op elk platform, maar er kunnen verschillen zijn in de implementatie omdat de systemen niet uniform zijn opgezet. 
Het is ook zo dat niet elk niveau in elk IMS-systeem aanwezig is. Daarom heeft elk IMS-systeem zijn eigen opstart- (en afsluitings-)dataset met de gedefinieerde niveaus.\\

Automation leest en interpreteert deze gegevens en voert de bijbehorende code uit volgens de instructies van het IMS-team. De code achter elk niveau wordt dus volledig beheerd door Automation. 
Deze bachelorproef bestaat uit het doorlopen van alle bestaande niveaus en het controleren of de bijbehorende code in Automation nog steeds voldoet aan de vereisten voor dat niveau. 
Zijn alle acties nog steeds relevant? Kunnen onnodige WAITS in de code worden verwijderd? Zijn er andere optimalisaties mogelijk? Het doel is om de start- en stopprocedures efficiënter te maken en de doorlooptijden 
van de verschillende flows te verkorten. Voor elk niveau moet ook duidelijk worden beschreven welke acties worden uitgevoerd. Hiervoor kan de bestaande documentatie als basis worden gebruikt.\\

%---------- Stand van zaken ---------------------------------------------------

\section{Literatuurstudie}
\label{sec:literatuurstudie}

% Hier beschrijf je de \emph{state-of-the-art} rondom je gekozen onderzoeksdomein, d.w.z.\ een inleidende, doorlopende tekst over het onderzoeksdomein van je bachelorproef. Je steunt daarbij heel sterk op de professionele \emph{vakliteratuur}, en niet zozeer op populariserende teksten voor een breed publiek. Wat is de huidige stand van zaken in dit domein, en wat zijn nog eventuele open vragen (die misschien de aanleiding waren tot je onderzoeksvraag!)?
% 
% Je mag de titel van deze sectie ook aanpassen (literatuurstudie, stand van zaken, enz.). Zijn er al gelijkaardige onderzoeken gevoerd? Wat concluderen ze? Wat is het verschil met jouw onderzoek?
% 
% Verwijs bij elke introductie van een term of bewering over het domein naar de vakliteratuur, bijvoorbeeld~\autocite{Hykes2013}! Denk zeker goed na welke werken je refereert en waarom.
% 
% Draag zorg voor correcte literatuurverwijzingen! Een bronvermelding hoort thuis \emph{binnen} de zin waar je je op die bron baseert, dus niet er buiten! Maak meteen een verwijzing als je gebruik maakt van een bron. Doe dit dus \emph{niet} aan het einde van een lange paragraaf. Baseer nooit teveel aansluitende tekst op eenzelfde bron.
% 
% Als je informatie over bronnen verzamelt in JabRef, zorg er dan voor dat alle nodige info aanwezig is om de bron terug te vinden (zoals uitvoerig besproken in de lessen Research Methods).

% Voor literatuurverwijzingen zijn er twee belangrijke commando's:
% \autocite{KEY} => (Auteur, jaartal) Gebruik dit als de naam van de auteur
%   geen onderdeel is van de zin.
% \textcite{KEY} => Auteur (jaartal)  Gebruik dit als de auteursnaam wel een
%   functie heeft in de zin (bv. ``Uit onderzoek door Doll & Hill (1954) bleek
%   ...'')

\textcite{malaika1994tale} states that IMS didn't have the performance nor fault tollerence needed for banking applications, particularly credit/debit transactions, which still are the most frequent transactions in these environments.
So two extensions were made: Fast Path and XRF (Extended Restart Facility).\\

\textcite{long2000ims} explains that there are two types of Fast Path online transactions.
\begin{itemize}
	\item \emph{Exclusive}: A load balancing group (LBG) is associated with each input message. A LBG is ready for balanced processing by one or more Fast Path programs. One LBG exists for each unique Fast Path message-driven application program.
	\item \emph{Potential}: These transactions are a mixture between standard IMS Full Function and Fast Path exclusive transactions. An exit is used to determine which instance is required for the message.
\end{itemize}
It also illustrates that XRF increases the availability of IMS systems. This is achieved by running a second IMS system on a seperate image, located on different physical machines. This is not a perfecct solution 
for everything as it comes with a couple downsides such as, not being able to protect against application errors or network outages.\\

The importance of high performance, high reliability systems is hammered home by \textcite{alex2012transaction}, where it is stated that these workloads are the most critical workloads run by an IT company.
A quote from Marriott International, mentioned in the book, estimates that the cost of their reservation system being unavailable would be three thousand dollars per second.
This is further enforced by \textcite{klein2011introduction}, which specifies that 90\% of the top worldwide companies in a large number of fields use IMS as their transaction management system. However these numbers are from 2011.



%---------- Methodologie ------------------------------------------------------
\section{Methodologie}
\label{sec:methodologie}

Hier beschrijf je hoe je van plan bent het onderzoek te voeren. Welke onderzoekstechniek ga je toepassen om elk van je onderzoeksvragen te beantwoorden? Gebruik je hiervoor literatuurstudie, interviews met belanghebbenden (bv.~voor requirements-analyse), experimenten, simulaties, vergelijkende studie, risico-analyse, PoC, \ldots?

Valt je onderwerp onder één van de typische soorten bachelorproeven die besproken zijn in de lessen Research Methods (bv.\ vergelijkende studie of risico-analyse)? Zorg er dan ook voor dat we duidelijk de verschillende stappen terug vinden die we verwachten in dit soort onderzoek!

Vermijd onderzoekstechnieken die geen objectieve, meetbare resultaten kunnen opleveren. Enquêtes, bijvoorbeeld, zijn voor een bachelorproef informatica meestal \textbf{niet geschikt}. De antwoorden zijn eerder meningen dan feiten en in de praktijk blijkt het ook bijzonder moeilijk om voldoende respondenten te vinden. Studenten die een enquête willen voeren, hebben meestal ook geen goede definitie van de populatie, waardoor ook niet kan aangetoond worden dat eventuele resultaten representatief zijn.

Uit dit onderdeel moet duidelijk naar voor komen dat je bachelorproef ook technisch voldoen\-de diepgang zal bevatten. Het zou niet kloppen als een bachelorproef informatica ook door bv.\ een student marketing zou kunnen uitgevoerd worden.

Je beschrijft ook al welke tools (hardware, software, diensten, \ldots) je denkt hiervoor te gebruiken of te ontwikkelen.

Probeer ook een tijdschatting te maken. Hoe lang zal je met elke fase van je onderzoek bezig zijn en wat zijn de concrete \emph{deliverables} in elke fase?

%---------- Verwachte resultaten ----------------------------------------------
\section{Verwacht resultaat, conclusie}
\label{sec:verwachte_resultaten}

Hier beschrijf je welke resultaten je verwacht. Als je metingen en simulaties uitvoert, kan je hier al mock-ups maken van de grafieken samen met de verwachte conclusies. Benoem zeker al je assen en de onderdelen van de grafiek die je gaat gebruiken. Dit zorgt ervoor dat je concreet weet welk soort data je moet verzamelen en hoe je die moet meten.

Wat heeft de doelgroep van je onderzoek aan het resultaat? Op welke manier zorgt jouw bachelorproef voor een meerwaarde?

Hier beschrijf je wat je verwacht uit je onderzoek, met de motivatie waarom. Het is \textbf{niet} erg indien uit je onderzoek andere resultaten en conclusies vloeien dan dat je hier beschrijft: het is dan juist interessant om te onderzoeken waarom jouw hypothesen niet overeenkomen met de resultaten.

