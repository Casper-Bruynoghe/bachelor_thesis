%---------- Inleiding ---------------------------------------------------------
\section{Inleiding}
\label{sec:inleiding}
In dit onderzoek wordt er specifiek gekeken naar het IMS subsysteem van de retail binnen KBC. Dit dient als een zien als een test omgeving voordat het kan uitgerold worden naar de rest van de IMS subsystemen.
Bij het opstarten van een IMS-subsysteem zijn er drie soorten die door IBM worden aangeboden: \emph{Warm start}, \emph{Cold start} en \emph{Emergency restart}. Binnen KBC heeft het Automation-team 
(subteam binnen Mainframe infrastructuur) uniforme start- en stopprocedures voorzien voor een groot aantal situaties. Er zijn ook een aantal gespecialiseerde procedures ontwikkeld die niet in de IBM-handleidingen te vinden zijn. 
Er moet dus een onderscheid worden gemaakt tussen wat IBM heeft voorzien – de drie hierboven beschreven soorten opstartprocedures – en wat via Automation in KBC is ingebouwd. 
Elke procedure is gebaseerd op het opstarten van het IMS-(sub)systeem, maar er worden ook aanvullende controles en commando's uitgevoerd, zoals het opstarten van regio's, het opstarten van IMS-verbindingen, enz.\\

Het IMS Infra-team is verantwoordelijk voor het beheer van het opstarten (en afsluiten) en dit is gebaseerd op een niveaustructuur. Elk niveau heeft een uniek nummer en voert een specifieke taak uit bij het opstarten. 
Elk niveau is functioneel hetzelfde in de verschillende IMS-systemen op elk platform, maar er kunnen verschillen zijn in de implementatie omdat de systemen niet uniform zijn opgezet. 
Het is ook zo dat niet elk niveau in elk IMS-systeem aanwezig is. Daarom heeft elk IMS-systeem zijn eigen opstart- (en afsluitings-) dataset met de gedefinieerde niveaus.\\

Automation leest en interpreteert deze gegevens en voert de bijbehorende code uit volgens de instructies van het IMS-team. De code achter elk niveau wordt dus volledig beheerd door Automation. 
Deze bachelorproef bestaat uit het doorlopen van alle bestaande niveaus en het controleren of de bijbehorende code in Automation nog steeds voldoet aan de vereisten voor dat niveau. 
Zijn alle acties nog steeds relevant? Kunnen onnodige WAITS in de code worden verwijderd? Zijn er andere optimalisaties mogelijk? Het doel is om de start- en stopprocedures efficiënter te maken en de doorlooptijden 
van de verschillende flows te verkorten. Voor elk niveau moet ook duidelijk worden beschreven welke acties worden uitgevoerd. Hiervoor kan de bestaande documentatie als basis worden gebruikt.\\

Als laatste heeft het onderzoek ook het doel om grondige documentatie te creëren over elke stap van de start- of stopprocedures en mogelijke optimalisaties dat geidentificeerd werden, zodat hiermee verder kan gewerkt worden
voor de rest van de IMS subsystemen.
%---------- Stand van zaken ---------------------------------------------------

\section{Literatuurstudie}
\label{sec:literatuurstudie}

% Hier beschrijf je de \emph{state-of-the-art} rondom je gekozen onderzoeksdomein, d.w.z.\ een inleidende, doorlopende tekst over het onderzoeksdomein van je bachelorproef. Je steunt daarbij heel sterk op de professionele \emph{vakliteratuur}, en niet zozeer op populariserende teksten voor een breed publiek. Wat is de huidige stand van zaken in dit domein, en wat zijn nog eventuele open vragen (die misschien de aanleiding waren tot je onderzoeksvraag!)?
% 
% Je mag de titel van deze sectie ook aanpassen (literatuurstudie, stand van zaken, enz.). Zijn er al gelijkaardige onderzoeken gevoerd? Wat concluderen ze? Wat is het verschil met jouw onderzoek?
% 
% Verwijs bij elke introductie van een term of bewering over het domein naar de vakliteratuur, bijvoorbeeld~\autocite{Hykes2013}! Denk zeker goed na welke werken je refereert en waarom.
% 
% Draag zorg voor correcte literatuurverwijzingen! Een bronvermelding hoort thuis \emph{binnen} de zin waar je je op die bron baseert, dus niet er buiten! Maak meteen een verwijzing als je gebruik maakt van een bron. Doe dit dus \emph{niet} aan het einde van een lange paragraaf. Baseer nooit teveel aansluitende tekst op eenzelfde bron.
% 
% Als je informatie over bronnen verzamelt in JabRef, zorg er dan voor dat alle nodige info aanwezig is om de bron terug te vinden (zoals uitvoerig besproken in de lessen Research Methods).

% Voor literatuurverwijzingen zijn er twee belangrijke commando's:
% \autocite{KEY} => (Auteur, jaartal) Gebruik dit als de naam van de auteur
%   geen onderdeel is van de zin.
% \textcite{KEY} => Auteur (jaartal)  Gebruik dit als de auteursnaam wel een
%   functie heeft in de zin (bv. ``Uit onderzoek door Doll & Hill (1954) bleek
%   ...'')
\textcite{alex2012transaction} definieert een transactie als een interactie met een informatiesysteem. Enkele voorbeelden hiervan zijn: het bijwerken van een winkelboekhouding, het goedkeuren van een creditcardtransactie of 
het verwijderen van een artikel uit de winkelvoorraad tijdens het afrekenen. Ook wordt het belang van deze systemen uiteengezet. Als een transactie niet wordt verwerkt, kan dit ertoe leiden dat een klant het dubbele
van het verschuldigde bedrag in rekening wordt gebracht. \\

\textcite{alex2012transaction}
\begin{itemize}
	\item OLTP (online transaction processing) vs batch processing) (p. 10)
\end{itemize}

Het belang van hoogwaardige, uiterst betrouwbare systemen wordt benadrukt door \textcite{alex2012transaction}, waarin wordt gesteld dat deze workloads de meest kritieke workloads zijn die door een IT-bedrijf worden uitgevoerd.
Een citaat van Marriott International, dat in het boek wordt genoemd, schat dat de kosten van het niet beschikbaar zijn van hun reserveringssysteem drieduizend dollar per seconde zouden bedragen.
Dit wordt verder onderstreept door \textcite{klein2011introduction}, waarin wordt gespecificeerd dat 90\% van de wereldwijde topbedrijven in een groot aantal sectoren IMS gebruiken als hun transactiebeheersysteem. 
Deze cijfers zijn echter afkomstig uit 2011.\\

Aangezien dit onderzoek gaat over het verbeteren van een systeem, is het van cruciaal belang om te weten hoe een dergelijke taak moet worden aangepakt. Een belangrijk punt dat in \textcite{amra2017modernizing} naar voren wordt gebracht, is dat iteratieve en incrementele processen, ook wel
de agile-aanpak genoemd, van cruciaal belang zijn bij het aanpakken van dit soort projecten. Het voorkomt dat teams overweldigd of gefrustreerd raken wanneer ze hun hele systeem in één keer volledig willen herzien.
Testen is een ander punt dat in dit boek wordt benadrukt. Het is de enige manier om ervoor te zorgen dat het systeem blijft werken nadat er onderdelen ervan zijn gewijzigd. Twee soorten tests vormen de pijlers van deze workflow:
\begin{itemize}
    \item Regressietesten: deze tests zorgen ervoor dat alle functionaliteiten naar verwachting werken wanneer slechts één onderdeel ervan wordt gewijzigd.
	\item Automatische unit-tests: met deze tests kan het ontwikkelingsteam kleine onderdelen van de applicatie testen, wat op zijn beurt een dieper begrip van de codebase vereist. Deze tests kunnen zo worden ingesteld dat ze automatisch worden uitgevoerd 
        na elke compilatie. Ze moeten daarom snel kunnen worden uitgevoerd om de ontwikkeling niet te hinderen.
\end{itemize}\\


%---------- Methodologie ------------------------------------------------------
\section{Methodologie}
\label{sec:methodologie}

% Hier beschrijf je hoe je van plan bent het onderzoek te voeren. Welke onderzoekstechniek ga je toepassen om elk van je onderzoeksvragen te beantwoorden? Gebruik je hiervoor literatuurstudie, interviews met belanghebbenden (bv.~voor requirements-analyse), experimenten, simulaties, vergelijkende studie, risico-analyse, PoC, \ldots?
%
% Valt je onderwerp onder één van de typische soorten bachelorproeven die besproken zijn in de lessen Research Methods (bv.\ vergelijkende studie of risico-analyse)? Zorg er dan ook voor dat we duidelijk de verschillende stappen terug vinden die we verwachten in dit soort onderzoek!
% 
% Vermijd onderzoekstechnieken die geen objectieve, meetbare resultaten kunnen opleveren. Enquêtes, bijvoorbeeld, zijn voor een bachelorproef informatica meestal \textbf{niet geschikt}. De antwoorden zijn eerder meningen dan feiten en in de praktijk blijkt het ook bijzonder moeilijk om voldoende respondenten te vinden. Studenten die een enquête willen voeren, hebben meestal ook geen goede definitie van de populatie, waardoor ook niet kan aangetoond worden dat eventuele resultaten representatief zijn.
% 
% Uit dit onderdeel moet duidelijk naar voor komen dat je bachelorproef ook technisch voldoen\-de diepgang zal bevatten. Het zou niet kloppen als een bachelorproef informatica ook door bv.\ een student marketing zou kunnen uitgevoerd worden.
% 
% Je beschrijft ook al welke tools (hardware, software, diensten, \ldots) je denkt hiervoor te gebruiken of te ontwikkelen.
% 
% Probeer ook een tijdschatting te maken. Hoe lang zal je met elke fase van je onderzoek bezig zijn en wat zijn de concrete \emph{deliverables} in elke fase?

\subsection{Overzicht creëren van huidige infrastructuur}
Er zal grondig onderzoek worden gedaan naar de staat van de huidige infrastructuur. Eerst zal er enkele weken tijd gespendeerd worden aan een opleiding over IMS en de infrastructuur binnen KBC, dan zal er samen
met de interne documentatie van KBC een high-level overzicht worden gemaakt. Dit zal een goed overzicht geven van de zwakke punten en mogelijke verbeterpunten van het huidige systeem.\\

Hierdoor zal ook duidelijk worden \textcolor{red}{welke externe componenten en afhankelijke systemen betrokken zijn bij de AUTOSTART en AUTOSTOP procedures}.
Het is van groot belang dat het bekend is hoe de verschillende systemen met elkaar communiceren, of de kennis over elk stuk software nog binnen het bedrijf 
aanwezig is en wanneer iemand er voor het laatst mee heeft gewerkt.

\subsection{Verzamelen van vereisten}
In deze fase is het belangrijk om volgende vragen te kunnen beantwoorden:
\begin{itemize}
	\item \textcolor{red}{Welke statistieken moeten worden bekeken worden om de performantie te evalueren?}
	\item \textcolor{red}{Welke monitoring tools zijn er aanwezig om de performantie van de start en stop procedures te analyseren?}
\end{itemize}

Voordat deze fase van start kan gaan moet er duidelijkheid zijn over de vereisten voor een haalbare oplossing. Om dit te bereiken, zal het onderzoek het probleem opsplitsen en alle vereisten verzamelen.\\

Eerst zal er een uitgebreide vergadering plaatsvinden met de betreffende medewerkers van KBC, waar de vereisten zullen worden besproken en eventueel gerangschikt van meest belangrijk naar minst belangrijk.\\

Ten tweede zal op basis van de resultaten van de eerste stap een fase volgen waarin de belangrijkste prestatiemaatstaven worden gedefinieerd. Bijvoorbeeld:
\begin{itemize}
	\item Percentage van verwijderde dode code.
	\item Afname van gemiddelde opstart-/afsluittijd.
\end{itemize}

Ten slotte wordt er een uitgebreide lijst opgesteld met alle vereisten waaraan de oplossing moet voldoen om als geschikt te worden beschouwd. Deze lijst wordt opgesteld volgens 
de \href{https://en.wikipedia.org/wiki/MoSCoW_method}{MoSCoW} methode, zodat er prioriteit kan worden aangegeven aan elke vereiste. Dit zorgt er voor dat het duidelijk is waar de meeste focus moet gelegd worden
tijdens het onderzoek.

\subsection{Literatuurstudie}
In deze fase van het onderzoek zal een literatuuronderzoek worden uitgebreid om inzicht te krijgen in wat er al bekend en vastgesteld is over het onderwerp. 
Dit omvat een overzicht van de belangrijkste concepten en bestaande technologieën. Deze technologieën en technieken zullen worden geanalyseerd om de projectvereisten 
duidelijker te definiëren en een duidelijker beeld te geven van de uiteindelijke oplossing. 
Uiteindelijk is dit onderzoek bedoeld om ervoor te zorgen dat geen belangrijke onderwerpen of relevante hulpmiddelen over het hoofd worden gezien, 
zodat het onderzoek kan worden voortgezet met een volledig en gedetailleerd inzicht. Deze fase zal de volledige onderzoeksperiode doorlopen.\\

Belangrijke zaken waar er onderzoek naar gedaan moet worden tijdens deze fase zijn:
\begin{itemize}
	\item \textcolor{red}{Welke onderhoudsstrategie op lange termijn kan voorkomen dat performanceverlies zich in de toekomst opnieuw voordoet?}
	\item \textcolor{red}{Welke performantieproblemen worden momenteel waargenomen tijdens het opstarten en afsluiten?}
	\item \textcolor{red}{Hoe kan dode code, verouderde scripts of ongebruikte onderdelen veilig worden geïdentificeerd en verwijderd?}
\end{itemize}

\subsection{Volledige lijst}
Het doel van deze fase is om een lange lijst op te stellen van geïdentificeerde problemen, inefficiënties en knelpunten.
Dit geeft een duidelijk overzicht van waar de grootste verbeteringen in de huidige architectuur kunnen worden aangebracht.\\ 

Deze lijst zal opgesteld worden met behulp van de geïdentificeerde vereisten in fase twee van het onderzoek. Hierdoor is het belangrijk dat het verzamelen 
van de vereisten grondig genoeg wordt uitgevoerd, zodat er niets over het hoofd wordt gezien.

\subsection{Korte lijst}
De korte lijst wordt samengesteld uit de meest veelbelovende items uit de volledige lijst, opgesteld in de voorgaande stap. De volledige lijst wordt zorgvuldig beoordeeld op basis van de volgende criteria:
\begin{itemize}
	\item Complexiteit: \textcolor{red}{is deze verandering haalbaar binnen de tijdspanne van dit onderzoek?}
	\item Voldoen aan vereisten: \textcolor{red}{is het, op basis van de vereisten die aan het begin zijn vastgesteld, zinvol om dit onderdeel van de infrastructuur te updaten?}
	\item Belang: \textcolor{red}{levert het tijd besteden aan het updaten van dit onderdeel van de infrastructuur een aanzienlijke prestatieverbetering op?}
\end{itemize}
Deze korte lijst wordt vervolgens voorgelegd aan het automatiseringsteam voor feedback.

\subsection{Proof of concept}
Er zal een proof-of-concept (PoC) worden gemaakt met de geselecteerde wijzigingen uit de korte lijst. Elk van deze wijzigingen zal worden getest volgens de vooraf gedefinieerde vereisten. 
Dit zal vervolgens worden voorgelegd aan het automatiseringsteam om feedback te verzamelen. Deze PoC zal worden geleverd met gegevens die verbeteringen ten opzichte van de vorige versie en in vergelijking met het huidige systeem aantonen.\\

De PoC zal verschillende iteraties doorlopen voordat er een geschikte oplossing wordt gevonden die kan worden geïmplementeerd. Hiervoor is het belangrijk om concreet na te denken over 
\textcolor{red}{hoe de huidige oplossing zich aansluit bij de best practices van IBM en de interne normen van KBC}.


%---------- Verwachte resultaten ----------------------------------------------
\section{Verwacht resultaat, conclusie}
\label{sec:verwachte_resultaten}
% 
% Hier beschrijf je welke resultaten je verwacht. Als je metingen en simulaties uitvoert, kan je hier al mock-ups maken van de grafieken samen met de verwachte conclusies. Benoem zeker al je assen en de onderdelen van de grafiek die je gaat gebruiken. Dit zorgt ervoor dat je concreet weet welk soort data je moet verzamelen en hoe je die moet meten.
% 
% Wat heeft de doelgroep van je onderzoek aan het resultaat? Op welke manier zorgt jouw bachelorproef voor een meerwaarde?
% 
% Hier beschrijf je wat je verwacht uit je onderzoek, met de motivatie waarom. Het is \textbf{niet} erg indien uit je onderzoek andere resultaten en conclusies vloeien dan dat je hier beschrijft: het is dan juist interessant om te onderzoeken waarom jouw hypothesen niet overeenkomen met de resultaten.
Dit onderzoek zal een aanzienlijke impact hebben binnen KBC, aangezien IBM-mainframes een abonnement vereisen op basis van de hoeveelheid CPU-tijd die gedurende een bepaalde periode wordt gebruikt. Als een veelgebruikt stuk software, zoals IMS,
wordt geoptimaliseerd en de CPU-tijd wordt verminderd, kan het bedrijf geld besparen en wordt het onderhoud van het systeem eenvoudiger voor de verschillende infrastructuurteams.\\

Er wordt een aanzienlijke verbetering verwacht in de tijd die een IMS-(sub)systeem nodig heeft om te starten of te stoppen. Dit omvat ook het verwijderen van dode code en het algemeen verbeteren van oudere processen.

